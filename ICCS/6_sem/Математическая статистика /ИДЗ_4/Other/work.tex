% Intended LaTeX compiler: xelatex
\documentclass[12pt]{spbstu-task}
\usepackage{amssymb}
\usepackage{amsmath}
\usepackage{xunicode}
\usepackage{fontspec}
\usepackage{graphicx}
\usepackage{longtable}
\usepackage{wrapfig}
\usepackage{rotating}
\usepackage[normalem]{ulem}
\usepackage{capt-of}
\usepackage{hyperref}
\institute{Институт компьютерных наук и кибербезопасности}
\school{Высшая школа технологии искусственного интелекта}
\program{Направление 02.03.01 Математика и компьютерные науки}
\subject{Математическая Статистика}
\reviewertitle{}\reviewer{Малов Сергей Васильевич}
\usepackage{array}
\usepackage{float}
\usepackage{romannum}
\setmainfont{Times New Roman}
\defaultfontfeatures{Ligatures=TeX,Scale=MatchLowercase,Mapping=tex-text}
\newcommand*{\myref}[2]{\hyperref[{#1}]{{#2}~\ref*{#1}}}
\newcommand{\hlabel}{\phantomsection\label}
\newcommand*{\fullref}[1]{\hyperref[{#1}]{\autoref*{#1} \nameref*{#1}}}
\setlength{\abovecaptionskip}{5pt}
\setlength{\belowcaptionskip}{0pt}
\hypersetup{hidelinks}
\newcommand{\overbar}[1]{\mkern 1.5mu\overline{\mkern-1.5mu#1\mkern-1.5mu}\mkern 1.5mu}
\usepackage{dsfont}
\usepackage[makeroom]{cancel}
\usepackage{tikz}
\usepackage{pgfplots}
\usepackage{siunitx}
\setlength{\textfloatsep}{0pt}
\setcounter{secnumdepth}{3}
\author{Михалец Мартин}
\date{2025}
\title{Индивидуальное домашнее задание №4\\\medskip
\large Вариант 21 (513020125)}
\hypersetup{
 pdfauthor={Михалец Мартин},
 pdftitle={Индивидуальное домашнее задание №4},
 pdfkeywords={},
 pdfsubject={},
 pdfcreator={Emacs 29.4 (Org mode 9.7.25)}, 
 pdflang={English}}
\usepackage{biblatex}
\addbibresource{~/documents/bib/biblio.bib}
\begin{document}

\sloppy
\pagenumbering{arabic}
Михалец Мартин 5130201/20102 \hfill ИДЗ №4 \hfill Вариант 21 (513020125)
\section{\textbf{Задание} 1:}
\label{sec:orgc069df6}
\begin{center}
\includegraphics[width=\textwidth]{./task1.png}
\end{center}
\subsection{Пункт a}
\label{sec:org12aca2d}
\begin{figure}[H]
\centering
\input{results}
\end{figure}

\textbf{Формулировка линейной регрессионной модели}

Линейная регрессионная модель зависимости \(Y\) от \(X\) имеет вид: \[
Y = \beta_1 + \beta_2 X + \epsilon, \] где:
\begin{itemize}
\item \(\beta_1\) --- параметр сдвига,
\item \(\beta_2\) --- параметр масштаба,
\item \(\epsilon\) --- случайная ошибка.
\end{itemize}

\textbf{Построение МНК-оценок параметров}

Метод наименьших квадратов (МНК) используется для нахождения оценок
\(\hat{\beta_1}\) и \(\hat{\beta_2}\), которые минимизируют сумму
квадратов остатков.

\begin{displaymath}
  \beta_1 = 11.0081,\quad
  \beta_2 = -0.0689,\quad
  R^2 \;\text{линейной модели:}\; 0.017.
\end{displaymath}

\begin{figure}[H]
\centering
\input{linear-regression}
\end{figure}

\textbf{Распределение точек относительно линии}

Точки разбросаны, линия не отражает тренд, что говорит о плохом
соответствии.

\textbf{Наклон линии:} Линия близка к горизонтальной, зависимость слабая.

Таким образом, Между \(X\) и \(Y\) нет линейной зависимости.  Линейная
модель не подходит для описания данных.
\subsection{Пункт b}
\label{sec:orgdaca360}
\textbf{Формулировка полиномиальной регрессионной модели}

Полиномиальная регрессионная модель зависимости \(Y\) от \(X\) имеет
вид: \[Y = \beta_1 + \beta_2 X + \beta_3 X^2 + \epsilon,\] где:
\begin{itemize}
\item \(\beta_1\) --- параметр сдвига,
\item \(\beta_2\) --- линейный коэффициент при \(X\),
\item \(\beta_3\) --- квадратичный коэффициент при \(X^2\),
\item \(\epsilon\) --- случайная ошибка
\end{itemize}

\begin{figure}[H]
\centering
\input{polynomial-regression}
\end{figure}

Полиномиальная модель:
\begin{displaymath}
  \beta_1 = 11.5086,\quad
  \beta_2 = -0.4751,\quad
  \beta_3 = 0.0469,\quad
  R^2 \text{полиномиальной модели:} 0.0724.
\end{displaymath}

\textbf{Распределение точек относительно линии:} Точки разбросаны, линия не
отражает тренд, что говорит о плохом соответствии.

\textbf{Низкий R\textsuperscript{2}} означает, что квадратичная модель плохо описывает связь
между \(X\) и \(Y\).

Результаты указывают на то, что квадратичная модель не подходит для
описания данных.
\subsection{Пункт c}
\label{sec:orge3d0a2b}
\begin{figure}[H]
\centering
\input{residuals-hist}
\end{figure}

\begin{figure}[H]
\centering
\input{residuals-qq}
\end{figure}

\textbf{Проверка нормальности с помощью критерия \(\chi^2\)}

Этапы:
\begin{enumerate}
\item Гипотезы:
\begin{itemize}
\item \(H_0\): Остатки имеют нормальное распределение.
\item \(H_1\): Остатки не имеют нормального распределения.
\end{itemize}
\item Разделить данные на интервалы (бины): Используем те же интервалы,
что и в гистограмме.
\item Рассчитать наблюдаемые (\(O_i\)) и ожидаемые (\(E_i\)) частоты:
\begin{itemize}
\item \(E_i = N \cdot P\) (для \(i\)-го интервала), где \(P\) ---
вероятность из нормального распределения \(N(\mu, \sigma^2)\).
\end{itemize}
\item Вычислить статистику \(\chi^2\): \[\chi^2 = \sum \frac{(O_i -
   E_i)^2}{E_i}.\]
\item Сравнить с критическим значением \(\chi^2\): Если \(\chi^2 >
   \chi^2_{\text{крит}}\), отвергаем \(H_0\).
\end{enumerate}

\begin{displaymath}
  \mathcal{X}^2 = 1.2502,\quad
  \text{Критическое значение}\; = 9.2103,\quad
  p\text{-value} = 0.5352.
\end{displaymath}

Не отвергаем \(H_0\): распределение нормальное

\textbf{Визуально:} Остатки близки к нормальному распределению.

\textbf{Статистически:} Критерий \(\chi^2\) не выявил значимых отклонений от
нормальности на уровне \(\alpha\) = 
0.01.

Предположение о нормальности ошибок выполняется.
\subsection{Пункт d}
\label{sec:orga54edb4}
Частные интервалы строятся для каждого параметра отдельно, используя
\(t\)-распределение.

\textbf{Формула:} \[\hat{\beta_j} \pm t_{1-\alpha/2, n-p} \cdot
SE(\hat{\beta_j}),\] где:
\begin{itemize}
\item \(\hat{\beta_j}\) --- оценка параметра,
\item \(SE(\hat{\beta_j})\) --- стандартная ошибка параметра,
\item \(t_{1-\alpha/2}\) --- критическое значение \(t\)-распределения,
\item \(n\) --- число наблюдений,
\item \(p\) --- число параметров модели (для квадратичной модели \(p =
  3\)).
\end{itemize}

\textbf{Доверительные интервалы (уровень 0.99):}
\begin{itemize}
\item Доверительный интервал для \(\beta_2\) (
99\%): [
-1.1556, 0.2054]
\item Доверительный интервал для \(\beta_3\) (
99\%): [
-0.0282, 0.1220]
\end{itemize}

\textbf{Совместные доверительные интервалы}

Совместные интервалы учитывают корреляцию между оценками параметров.
Используем метод Бонферрони или \(F\)-распределение.

\textbf{Метод Бонферрони}

Формула: \[\hat{\beta_j} \pm t_{1-\alpha/(2k),n-p} \cdot
SE(\hat{\beta_j}),\] где \(k = 2\) (число параметров \(\beta_2\) и
\(\beta_3\)).

\begin{figure}[H]
\centering
\input{combined-distribution-interval}
\end{figure}

Ковариационная матрица для \(\beta_2\) и \(\beta_3\):
\begin{pmatrix}
  0.0643 & -0.0068 \\
  -0.0068 &  0.0008 \\
\end{pmatrix}

Совместные интервалы (Бонферрони):
\begin{itemize}
\item \(\beta_2\): [-1.2218, 0.2716]
\item \(\beta_3\): [-0.0282, 0.1220]
\end{itemize}

\textbf{Метод F-распределения}

Формула: \[(\hat{\beta} - \beta)^T \cdot Cov(\hat{\beta})^{-1} \cdot
(\hat{\beta} - \beta) \leq F_{1-\alpha, 2, n-p},\] где \(F_{1-\alpha,
2, n-p}\) --- критическое значение \(F\)-распределения.

Полная ковариационная матрица:
\begin{pmatrix}
  0.2129 & -0.0935 &  0.0084 \\
  -0.0935 &  0.0643 & -0.0068 \\
   0.0084 & -0.0068 &  0.0008 \\
\end{pmatrix}

Вектор оценок параметров \([\beta_2, \beta_3]\): [
  -0.4751, 0.0469]
\subsection{Пункт e}
\label{sec:orgaf35a36}
\textbf{Гипотеза линейности}
\begin{itemize}
\item \(H_0\): Зависимость \(Y\) от \(X\) линейна (\(\beta_3 = 0\)).
\item \(H_1\): Зависимость нелинейна (\(\beta_3 \neq 0\)).
\end{itemize}

\textbf{Гипотеза независимости}
\begin{itemize}
\item \(H_0\): \(Y\) не зависит от \(X\) (\(\beta_2 = \beta_3 = 0\)).
\item \(H_1\): \(Y\) зависит от \(X\) (хотя бы один из \(\beta_2, \beta_3
  \neq 0\)).
\end{itemize}

\textbf{Проверка гипотезы линейности (\(H_0: \beta_3 = 0\)):}
\begin{itemize}
\item \(t\)-статистика: 1.6758
\item \(p\)-значение:
0.1004
\item Нет оснований отвергать гипотезу о линейности (\(p > 0.01\)).
\end{itemize}

\textbf{Проверка гипотезы независимости (\(H_0: \beta_2 = 0\)):}
\begin{itemize}
\item \(t\)-статистика: -1.8741
\item \(p\)-значение: 0.0671
\item Нет оснований отвергать гипотезу о независимости (\(p > 0.01\)).
\end{itemize}
\subsection{Пункт f}
\label{sec:orgb3cb446}
Сравнение моделей по AIC и BIC:
\begin{table}[H]
  \centering
  \begin{tabular}{lrr} \toprule
    Модель & AIC & BIC \\ \midrule
    Линейная & 186.2028 & 191.9388 \\
    Квадратная & 185.3011 & 192.9492 \\ \bottomrule
  \end{tabular}
\end{table}

Так как разница в AIC и BIC небольшая, можно сказать, что обе модели
адекватны, но:
\begin{itemize}
\item AIC слегка за квадратную,
\item BIC слегка за линейную.
\end{itemize}
\subsection{Пункт g}
\label{sec:org7a97cb4}
\textbf{Характер зависимости \(Y\) от \(X\)}
\begin{itemize}
\item \textbf{Линейная модель:} \[Y = 11.01 + (-0.07)X,\quad R^2 = 0.017.\]
\begin{itemize}
\item Крайне низкий \(R^2\) (
1.7\%) указывает на отсутствие линейной
зависимости.
\item Коэффициент \(\beta_2 = -0.07\) статистически незначим (доверительный интервал

\([-0.22, 0.08]\) включает ноль).
\end{itemize}
\item \textbf{Квадратичная модель:} \[Y = 11.51 + (-0.48)X + (0.05)X^2,\quad R^2 = 0.0724.\]
\begin{itemize}
\item \(R^2 = 7.2\%\) показывает,
что модель объясняет лишь незначительную часть вариации.
\item Коэффициенты:
\begin{itemize}
\item \(\beta_2 = -0.48\) (линейный
член): интервал 
\([-0.99, 0.03]\) включает ноль.
\item \(\beta_3 = 0.05\)
(квадратичный член): интервал 
\([-0.01, 0.1]\) включает ноль.
\end{itemize}
\end{itemize}
\end{itemize}

\textbf{Проверка гипотез}

Остатки близки к нормальному распределению.  Критерий \(\chi^2\) не
выявил значимых отклонений от нормальности на уровне

\(\alpha = 0.01\).

\emph{Предположение о нормальности ошибок выполняется.}

\textbf{AIC/BIC}
\begin{table}[H]
  \centering
  \begin{tabular}{lrr} \toprule
    Модель & AIC & BIC \\ \midrule
    Линейная & 186.2028 & 191.9388 \\
    Квадратная & 185.3011 & 192.9492 \\ \bottomrule
  \end{tabular}
\end{table}

\begin{itemize}
\item \textbf{Линейная модель} имеет более низкий BIC, чем квадратичная.
\item \textbf{Квадратная модель} имеет более низкий AIC, чем линейная.
\end{itemize}

\textbf{Аномалии в результатах}
\begin{itemize}
\item \textbf{Парадокс низкого \(R^2\):}
\begin{itemize}
\item Линейная модель объясняет менее 3\% вариации, что ставит под
сомнение её практическую применимость.
\end{itemize}
\end{itemize}

\textbf{Итоговый вывод}
\begin{itemize}
\item \textbf{Отсутствие значимой связи:} Ни линейная, ни квадратичная модели не
демонстрируют статистически значимой зависимости \(Y\) от \(X\) на
уровне 
\(\alpha = 0.01\).
\item \textbf{Рекомендации:}
\begin{itemize}
\item Проверить данные на наличие выбросов или ошибок.
\item Рассмотреть другие предикторы или преобразования.
\item Увеличить объем данных для повышения надежности тестов.
\end{itemize}
\end{itemize}
\section{\textbf{Задание} 2:}
\label{sec:org69c4c03}
\begin{center}
\includegraphics[width=\textwidth]{./task2.png}
\end{center}
\subsection{Пункт a}
\label{sec:orgc954c3a}
\subsubsection{Формулировка модели двухфакторного дисперсионного анализа}
\label{sec:org86aff1a}
Модель с взаимодействием факторов: \[Y_{ijk} = \mu + \alpha_i +
\beta_j + (\alpha \beta)_{ij} + \epsilon_{ijk},\] где:
\begin{itemize}
\item \(Y_{ijk}\) --- наблюдаемое значение переменной \(Y\) для \(i\)-го
уровня фактора \(A\), \(j\)-го уровня фактора \(B\), \(k\)-го
повторения,
\item \(\mu\) --- общее среднее,
\item \(\alpha_i\) --- эффект \(i\)-го уровня фактора \(A\),
\item \(\beta_j\) --- эффект \(j\)-го уровня фактора \(B\),
\item \((\alpha \beta)_{ij}\) --- эффект взаимодействия факторов \(A\) и
\(B\),
\item \(\epsilon_{ijk} \sim N(0, \sigma^2)\) --- случайная ошибка.
\end{itemize}
\subsubsection{Построение МНК-оценок параметров}
\label{sec:orga0bcff0}
Оценки параметров полной модели:
\begin{verbatim}
(Intercept)  17.7160
a2           -1.8700
a3            0.6380
b2           -0.4720
b3           -2.0120
a2:b2        -0.3160
a3:b2        -0.5920
a2:b3        -2.0720
a3:b3        -0.2560
\end{verbatim}
\subsubsection{Несмещенная оценка дисперсии}
\label{sec:org9861127}
Несмещенная оценка дисперсии ошибок: 
\[\hat{\sigma}^2 = \frac{SS_{\text{res}}}{df_{\text{res}}} = 1.5799,\] где:
\begin{itemize}
\item \(SS_{\text{res}}\) --- сумма квадратов остатков,
\item \(df_{\text{res}} = n - p\) --- степени свободы (\(n\) --- число
наблюдений, \(p\) --- число параметров).
\end{itemize}
\subsection{Пункт b}
\label{sec:org071f9e8}
Сводная таблица средних значений \(Y\):
\begin{table}[H]
  \centering
  \begin{tabular}{c|ccc} \toprule
    $A \backslash B$ & 1 & 2 & 3 \\ \midrule
    1 & 17.716 & 17.244 & 15.704 \\
    2 & 15.846 & 15.058 & 11.762 \\
    3 & 18.354 & 17.290 & 16.086 \\ \bottomrule
  \end{tabular}
\end{table}

\begin{figure}[H]
\centering
\input{dependency}
\end{figure}

\textbf{Визуальная проверка аддитивности:}
\begin{itemize}
\item Основная часть графика (особенно \(B = 1\) и \(B = 2\)) говорит в
пользу аддитивной модели, так как линии почти параллельны.
\item Однако при \(B = 3\) отклонения заметнее, поэтому взаимодействие не
исключено.
\end{itemize}
\subsection{Пункт c}
\label{sec:org6a23c9c}
\begin{figure}[H]
\centering
\input{residuals-2}
\end{figure}

\begin{figure}[H]
\centering
\input{qq-2}
\end{figure}

\textbf{Тест Шапиро-Уилка:} p-value = 
0.7283

\textbf{Не отвергаем \(H_0\):} p-value > \(\alpha\) = 0.2.

\textbf{Результаты:}
\begin{itemize}
\item Гистограмма: Распределение остатков близко к нормальному, совпадает
с наложенной кривой \(N(0, \sigma^2)\).
\item Q-Q график: Точки лежат вдоль линии \(y = x\), что подтверждает
нормальность.
\item Тест Шапиро-Уилка: гипотеза о нормальности не отвергается.
\end{itemize}
\subsection{Пункт d}
\label{sec:org3539406}
Таблица ANOVA:
\begin{table}[H]
  \centering
  \begin{tabular}{l|ccccc} \toprule
    & Df & Sum Sq & Mean Sq & F value & Pr(>F) \\ \midrule
    a & 2 & 81.81136 & 40.90568 & \num{25.89082} & \num{1.07691e-07} \\
    b & 2 & 62.13282 & 31.06641 & \num{19.66316} & \num{1.69202e-06} \\
    a:b & 4 & 8.75332 & 2.18833 & \num{1.38508} & \num{2.58557e-01} \\
    Residuals & 36 & 56.87748 & 1.57993 & \num{NA} & \num{NA} \\ \bottomrule
  \end{tabular}
\end{table}

\textbf{Результаты ANOVA}
\begin{itemize}
\item Фактор \(A\): \[F = 25.89,\ p\text{-value} < 0.2 \ \rightarrow \
  \text{значимо влияет на } Y.\]
\item Фактор \(B\): \[F = 19.66,\ p\text{-value} < 0.2 \ \rightarrow \
  \text{значимо влияет на } Y.\]
\item Взаимодействие \(A \times B\): \[F = 1.38,\ p\text{-value} > 0.2 \
  \rightarrow \ \text{незначимо влияет на } Y.\]
\item Вывод: На уровне значимости \(\alpha = 0.20\) факторы \(A\) и \(B\)
значимо влияют на отклик \(Y\), поскольку соответствующие p-значения
меньше 0.20.  В то же время, взаимодействие факторов \(A \times B\)
незначимо (\(p > 0.2\)).  Это означает, что влияние каждого из
факторов на \(Y\) не зависит от уровня другого фактора.  Это говорит
о том, что можно использовать аддитивную модель без взаимодействия.
\end{itemize}
\subsection{Пункт e}
\label{sec:org9e4ae44}
Для выбора оптимальной модели используются критерии:
\begin{itemize}
\item AIC оценивает баланс между качеством подгонки модели и её
сложностью, накладывая штраф за избыточное количество параметров.
\item BIC работает аналогично AIC, но применяет более строгий штраф за
сложность, особенно при больших объемах данных.
\end{itemize}

Сравниваем две модели:
\begin{enumerate}
\item Полная модель (с взаимодействием): \[Y \sim A + B + A : B.\]
\item Аддитивная модель (без взаимодействия): \[Y \sim A + B.\]
\end{enumerate}

\begin{table}[H]
  \centering
  \begin{tabular}{lrr} \toprule
    Модель & AIC & BIC \\ \midrule
    Полная & 158.2451 & 176.3118 \\
    Аддитивная & 156.6867 & 167.5267 \\ \bottomrule
  \end{tabular}
\end{table}

\textbf{Вывод о сравнении моделей}

\begin{itemize}
\item \textbf{Результаты AIC и BIC:}
\begin{itemize}
\item Полная модель имеет AIC = 158.2451, в то время как аддитивная
модель имеет AIC = 156.6867.  Это указывает на малое преимущество
аддитивной модели.
\item Полная модель также имеет BIC = 176.3118, а аддитивная модель ---
BIC = 167.5267.  Разница подтверждает выбор аддитивной модели.
\end{itemize}
\item \textbf{Заключение:}
\begin{itemize}
\item Аддитивная модель \textbf{предпочтительнее}, так как она лучше соответствует
данным, что подтверждается меньшими значениями AIC и BIC.
\item Взаимодействие оказалось незначимым.
\end{itemize}
\end{itemize}
\subsection{Пункт f}
\label{sec:org3c84685}
\subsubsection{Основные эффекты факторов A и B}
\label{sec:org4a3e25d}
\begin{itemize}
\item \textbf{Фактор A:} Оказал сильное статистически значимое влияние на \(Y\)
(\(F = 25.89, p < 0.2\)).
\item \textbf{Фактор B:} Также значимо влияет на \(Y\) (\(F = 19.66, p < 0.2\)).
\end{itemize}
\subsubsection{Взаимодействие факторов \(A \times B\)}
\label{sec:org9af8c53}
\begin{itemize}
\item \textbf{Статистическая значимость:} Взаимодействие значимо (\(F = 1.38, p <
  0.2\)).
\item \textbf{Визуальное подтверждение:} График зависимости \(Y\) от \(A\) при
фиксированных \(B\) показывает отсутствие пересечение линий, причём
они почти параллельны, что указывает на аддитивность эффектов.
\end{itemize}
\subsubsection{Выбор оптимальной модели}
\label{sec:org980c163}
AIC/BIC:
\begin{table}[H]
  \centering
  \begin{tabular}{lrr} \toprule
    Модель & AIC & BIC \\ \midrule
    Полная & 158.2451 & 176.3118 \\
    Аддитивная & 156.6867 & 167.5267 \\ \bottomrule
  \end{tabular}
\end{table}

Разница \(\Delta AIC = -1.55\) и \(\Delta BIC = -8.78\) указывает на
преимущество аддитивной модели.

Полная модель ложно учитывает взаимодействия, они не дают реального
вклада, но увеличивают размерность и ухудшают обобщающую способность
\subsubsection{Нормальность остатков}
\label{sec:orgb6b0507}
\begin{itemize}
\item Тест Шапиро-Уилка: \[p\text{-value} = 0.7283 \implies \text{гипотеза
  о нормальности остатков не отвергается}.\]
\item Графическая проверка: Гистограмма остатков близка к нормальной
форме.
\item Q-Q график показывает совпадение точек с линией \(y = x\).
\end{itemize}

\textbf{Рекомендации:} Для прогнозирования \(Y\) лучше не учитывать
взаимодействие \(A \times B\), так как оно лишь усложняет модель.

\textbf{Итоговый вывод}
\begin{enumerate}
\item Аддитивная модель без взаимодействий предпочтительна по критериям
AIC/BIC и объясняет данные лучше полной.
\item Нормальность остатков подтверждена тестами и графиками.
\end{enumerate}

\textbf{Рекомендации:}
\begin{itemize}
\item Проверить данные на наличие выбросов для \(A=2\) и \(B=3\).
\item Использовать аддитивную модель для прогнозирования и анализа
эффектов.
\end{itemize}
\end{document}
